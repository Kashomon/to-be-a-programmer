\documentclass{article}
\usepackage[margin=0.5in]{geometry}

\usepackage{hyperref}
\setlength{\parindent}{0em}
% \setlength{\parskip}{1em}

\title{So You Want to be a Programmer}
\author{Josh Hoak \\ \texttt{joshuarhoak@gmail.com} \\\texttt{github.com/Kashomon/to-be-a-programmer}}
\date{Updated: \today}
\begin{document}
\maketitle

So, you want to be programmer? This guide is intended to give you some places to start!

\section*{Getting Started}

\begin{itemize}
\item \textit{Get a Computer!}: First you need a computer! It doesn't
necessarily need to be super fast, but you do need your own. Ideally, it would
be installed with Linux or MacOS operating systems, but Windows is ok too.

There are lots of places that sell refurbished computers for cheap. For
example, check out Green PC in Tacoma, InterConnection in Seattle, and FreeGeek
in Seattle.
\item \textit{Learn the Tools}: Programmers must develop expertise at both coding and the tools they use.
  \begin{itemize}
  \item Choose a Language. Python, JavaScript, Go, Ruby, and Java are all good
  options. Prefer a language that's commonly used
  \item Choose an editor. I use Vim, but there are lots of awesome editors out
  there including Emacs, Sublime, IntelliJ, etc.  Prefer an editor that is
  capable at many different languages.
  \item Learn about version control and choose one for your projects. The most
  common is Git followed by Mercurial. Try using Github (\url{https://github.com})
  or BitBucket \url{https://bitbucket.org/}.
  \end{itemize}
\item \textit{Read}: Reddit's guide to learning how to program: \url{https://www.reddit.com/r/learnprogramming/wiki/faq#wiki_how_do_i_get_started_with_programming}
\item \textit{Read}: Teach yourself to Program in 10 Years by Peter Norvig: \url{http://norvig.com/21-days.html}
\item \textit{Read}: The posts in the Learning to Program topic on Quora: \url{https://www.quora.com/topic/Learning-to-Program}
\item \textit{Read}: Erik Trautman's "Why is Learning to Code so Damn Hard": \url{https://www.vikingcodeschool.com/posts/why-learning-to-code-is-so-damn-hard}
\item \textit{Online Schools}: There are lots online schools out there that will help get you started.
  \begin{itemize}
  \item \textit{MIT Open Course Ware} has dozens of excellent free lectures on Computer Science. \url{https://ocw.mit.edu/courses/electrical-engineering-and-computer-science/}
  \item \textit{CodeAcademy} (Mostly Free) has tons of awesome lessons and interactive tutorials. \url{https://www.codecademy.com/learn/all}
  \end{itemize}
\end{itemize}

\section*{What Next?}

At this point, you're getting comfortable doing simple tasks with your
programming language, you have a go-to editor, and you maybe even have some
code on Github. Not only that, but you've got a good idea about what the whole
programming thing is about and maybe even ideas about how you would put
together something that could actually be useful, like a web site or mobile
app! \textbf{What next?}

\begin{itemize}
  \item \textit{Practice}: Keep honing your skills! Try doing programming
  competitions and practice problems at CodeChef, TopCoder, and Google CodeJam.
  \item \textit{Find a Project!}: Find a project that sparks your interest! It
  could be a hobby probject, or app, or a website.
  \item \textit{Find People!}: It's always more fun and rewarding to work with
  a team. If you're still in school, check out clubs in your university. If
  you're out of school, try Meetups, or Facebook groups.
\end{itemize}

\end{document}
